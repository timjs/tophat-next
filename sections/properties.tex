% !TEX root=../main.tex

\section{Properties}
\label{sec:properties}


\subsection{Type preservation}
\label{sub:preservation}

\begin{theorem}[Type preservation under evaluation]
  For all expressions $e$
  such that $\Gamma,\Sigma \infers e:\tau$,
  if $e\eval v$,
  then $\Gamma,\Sigma \infers v:\tau$.
  \label{thm:pres-eval}
\end{theorem}



\begin{theorem}[Type preservation under normalisation]
  For all expressions $t$ and states $\sigma$
  such that $\Gamma,\Sigma \infers t:\Task\tau$ and $\Gamma,\Sigma \infers \sigma$,
  if $t,\sigma \normalise t',\sigma',\delta'$,
  then $\Gamma,\Sigma \infers t':\Task\tau$ and $\Gamma,\Sigma \infers \sigma'$.
  \label{thm:pres-norm}
\end{theorem}

\noindent
Where $\Gamma,\Sigma \infers \sigma$ means that for all $l\in \sigma$, it holds that
$\Gamma,\Sigma\infers \sigma(l):\Sigma(l)$.\fixme{can we remove $\Gamma$ from this relation? I think so...}


\begin{theorem}[Type preservation under fixing]
  For all expressions $t$ and states $\sigma$
  such that $\Gamma,\Sigma \infers t:\Task\tau$ and $\Gamma,\Sigma \infers \sigma$,
  if $t,\sigma,\delta \fix t',\sigma'$,
  then $\Gamma,\Sigma \infers t':\Task\tau$ and $\Gamma,\Sigma \infers \sigma'$.
  \label{thm:pres-fix}
\end{theorem}

\begin{theorem}[Type preservation under handling]
  For all expressions $t$, states $\sigma$ and inputs $i$
  such that $\Gamma,\Sigma \infers t:\Task\tau$ and $\Gamma,\Sigma \infers \sigma$,
  if $ t,\sigma \handle{i} t',\sigma',\delta'$,
  then $\Gamma,\Sigma\infers t':\Task\tau$ and $\Gamma,\Sigma\infers \sigma'$.
  \label{thm:pres-handle}
\end{theorem}

\begin{theorem}[Type preservation under interaction]
  For all expressions $e$, states $\sigma$ and inputs $i$
  such that $\Gamma,\Sigma \infers e:\Task\tau$ and $\Gamma,\Sigma \infers \sigma$,
  if $ e,\sigma \interact{i} e',\sigma'$,
  then $\Gamma,\Sigma\infers e':\Task\tau$ and $\Gamma,\Sigma\infers \sigma'$.
  \label{thm:pres-interaction}
\end{theorem}

\subsection{Progress}

A well-typed term of a task type is guaranteed to progress after normalisation,
unless it is failing.

We define what we mean with progress in \cref{thm:prog-interaction}.
\begin{theorem}[Progress under handling]
  %For all well typed expressions $e$ and states $\sigma$,
  For all expressions $t$ and states $\sigma$
  such that $\Gamma,\Sigma \infers t:\Task\tau$ and $\Gamma\Sigma \infers \sigma$,
  if $t,\sigma \normalise t',\sigma'$,
  then either $\Failing(t', \sigma')$
  or there exist $t''$, $\sigma''$, and $i$ such that $t',\sigma'\interact{i} t'',\sigma''$.
  \label{thm:prog-interaction}
\end{theorem}

%Where a well typed expression $e$ means that $\Gamma,\Sigma\infers e:\tau$ for
%some type $\tau$, and a well typed state means that $\Sigma\infers \sigma$.


\begin{lemma}[Failing means no interaction possible]
  For all expressions $t$
  such that $\Gamma,\Sigma\infers t:\Task\tau$,
  we have $\Failing(t)$ if and only if there exists no input $i$ such that $t,\sigma\interact{i} t',\sigma'$, for all $\sigma$ with $\Gamma,\Sigma\infers\sigma$ and a certain $t'$ and $\sigma'$.
  \label{lem:failing}
\end{lemma}


\begin{proof}
  Given $\Gamma,\Sigma\infers t:\Task\tau$ and $\Gamma,\Sigma\infers\sigma$ and after normalisation $t,\sigma\normalise t',\sigma'$, we find ourselves in either one of the following situations:

  There exists an $i$ such that $t',\sigma'\interact{i} t'',\sigma''$.

  There does not exist an $i$ such that $t',\sigma'\interact{i} t'',\sigma''$. In this case, we
  know that $\Failing(t',\sigma')$ must be true, by \cref{lem:failing}.

\end{proof}

\subsection{Fixing semantics is big-step}

\begin{lemma}
    For all well typed expressions $t$ and states $\sigma$,
    $t,\sigma,\delta\fix t',\sigma'$ and $t',\sigma'\normalise t'',\sigma''$,
    we have $t'=t''$ and $\sigma'=\sigma''$.
\end{lemma}
\fixme{I'm not sure that this property is interesting or needed...}
\subsection{Soundness and Completeness of Inputs}

In order to validate the function that calculates all possible inputs $\Inputs$,
we want to show that the set of possible inputs it produces is both sound and complete with respect to the handle semantics.
By sound we mean that all inputs in the set of possible inputs can actually be handled by the handle semantics,
and by complete we mean that the set of possible inputs contains all inputs that can be handled by the handle semantics.
\Cref{thm:safety-i} expresses exactly this property.

\begin{theorem}[Inputs function is sound and complete]
  For all tasks $t$, states $\sigma$, and inputs $i$
  such that $\Gamma,\Sigma \infers t:\Task\tau$ and $\Gamma,\Sigma \infers \sigma$,
  we have that $i \in \Inputs{(t,\sigma)}$ if and only if $t,\sigma \interact{i} t',\sigma'$.
  \label{thm:safety-i}
\end{theorem}

We prove the above theorem by induction over $e$. The proof is listed in the
appendix.
