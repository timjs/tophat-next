% !TEX root=../main.tex

\section{Proofs}

\subsection{\cref{thm:pres-eval}}

\begin{proof}
  We prove \cref{thm:pres-eval} by induction on $e$:

  \case
    {$e=v,e_1 e_2,\If{e_1}{e_2}{e_3},\tuple{e_1, e_2},\Fst\tuple{e_1, e_2},\Snd\tuple{e_1, e_2}, e_1 :: e_2, \Head e, \Tail e,\Share e, e_1 := e_2$}
    {Preservation has been proven for these cases by Pierce~\cite{books/Pierce02TAPL}.}

  % \case
  %   {$\userule{E-Edit}$}
  %   {Given that $\Gamma,\Sigma\infers\Edit e:\Task \tau$ and $\Gamma,\Sigma\infers s$,\refrule{T-Edit} gives us that $\Gamma,\Sigma\infers e:\tau$.
  %   The induction hypothesis gives us that $e,s\evaluate v,s'$ also preserves, and thus $\Gamma,\Sigma\infers v:\tau$ and $\Gamma,\Sigma\infers s'$.
  %   Therefore $\Gamma,\Sigma\infers\Edit v:\Task\tau$.}
  %
  % \case
  %   {$\userule{E-Enter}$}
  %   {Evaluation does not alter $e$ and $s$, therefore this case holds trivially.}
  %
  % \case
  %   {$\userule{E-Update}$}
  %   {Given that $\Gamma,\Sigma\infers \Edit e:\Task \tau$ and $\Gamma,\Sigma\infers s$, \refrule{T-Update} gives us that $\Gamma,\Sigma\infers e:\Ref \tau$.
  %   The induction hypothesis gives us that $e,s\evaluate l,s'$ also preserves, and thus $\Gamma,\Sigma\infers l:\Ref\tau$ and $\Gamma,\Sigma\infers s'$.
  %   Therefore $\Gamma,\Sigma\infers\Update l:\Task\tau$.}
  %
  % \case
  %   {$\userule{E-Fail}$}
  %   {Evaluation does not alter $e$ and $s$, therefore this case holds trivially.}
  %
  % \case
  %   {$\userule{E-Then}$}
  %   {Given that $\Gamma,\Sigma\infers e_1\Then e_2:\Task \tau$ and $\Gamma,\Sigma\infers s$, \refrule{T-Then} gives us that $\Gamma,\Sigma\infers e_1:\Task\tau_1$ and $\Gamma,\Sigma\infers e_2:\tau_1 \to \Task \tau$.
  %   By the induction hypothesis, we know that $e_1,s\evaluate t_1,s'$ preserves and thus $\Gamma,\Sigma\infers t_1:\Task\tau_1$ and $\Gamma,\Sigma\infers s'$.
  %   Therefore $\Gamma,\Sigma\infers t_1\Then e_2:\Task\tau$.}
  %
  % \case
  %   {$\userule{E-Next}$}
  %   {Given that $\Gamma,\Sigma\infers e_1\Next e_2:\Task \tau$ and $\Gamma,\Sigma\infers s$, \refrule{T-Next} gives us that $\Gamma,\Sigma\infers e_1:\Task\tau_1$ and $\Gamma,\Sigma\infers e_2:\tau_1 \to \Task \tau$.
  %   By the induction hypothesis, we know that $e_1,s\evaluate t_1,s'$ preserves and thus $\Gamma,\Sigma\infers t_1:\Task\tau_1$ and $\Gamma,\Sigma\infers s'$.
  %   Therefore $\Gamma,\Sigma\infers t_1\Next e_2:\Task\tau$.}
  %
  % \case
  %   {$\userule{E-And}$}
  %   {Given that $\Gamma,\Sigma\infers e_1\And e_2:\Task(\tau_1\times\tau_2)$ and $\Gamma,\Sigma\infers s$, \refrule{T-And} gives us that $\Gamma,\Sigma\infers e_1:\Task\tau_1$ and $\Gamma,\Sigma\infers e_2:\Task\tau_2$.
  %   By the induction hypothesis, we know that both $e_1,s\evaluate t_1,s'$ and $e_2,s'\evaluate t_2,s''$ preserve and thus $\Gamma,\Sigma\infers t_1:\Task\tau_1$, $\Gamma,\Sigma\infers s'$, $\Gamma,\Sigma\infers t_2:\Task\tau_2$ and $\Gamma,\Sigma\infers s''$.
  %   Therefore $\Gamma,\Sigma\infers t_1\And t_2:\Task(\tau_1\times\tau_2)$.}
  %
  % \case
  %   {$\userule{E-Or}$}
  %   {Given that $\Gamma,\Sigma\infers e_1\Or e_2:\Task\tau$ and $\Gamma,\Sigma\infers s$, \refrule{T-Or} gives us that $\Gamma,\Sigma\infers e_1:\Task\tau$ and $\Gamma,\Sigma\infers e_2:\Task\tau$.
  %   By the induction hypothesis, we have that both $e_1,s\evaluate t_1,s'$ and $e_2,s'\evaluate t_2,s''$ preserve and thus $\Gamma,\Sigma\infers t_1:\Task\tau$, $\Gamma,\Sigma\infers s'$, $\Gamma,\Sigma\infers t_2:\Task\tau$ and $\Gamma,\Sigma\infers s''$.
  %   Therefore $\Gamma,\Sigma\infers t_1\Or t_2:\Task\tau$.}
  %
  % \case
  %   {$\userule{E-Xor}$}
  %   {Evaluation does not alter $e$ and $s$, therefore this case holds trivially.}
  %
  % % \noindent\textbf{Case} $\userule{E-Appoint}$
  % %     Given that $\Gamma,\Sigma\infers u \At e:\Task\tau$ and $\Sigma\infers s$, the induction hypothesis gives us that $e,s\evaluate t,s'$ also preserves, and therefore by \refrule{T-Appoint} $\Gamma,\Sigma\infers u \At t:\Task\tau$.
\end{proof}
