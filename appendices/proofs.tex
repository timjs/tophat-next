% !TEX root=../main.tex

\section{Proofs}

\subsection{\cref{thm:pres-eval}}

\begin{proof}
  We prove \cref{thm:pres-eval} by induction on $e$:

  \case
    {$e=v,e_1 e_2,\If{e_1}{e_2}{e_3},\tuple{e_1, e_2},\Fst\tuple{e_1, e_2},\Snd\tuple{e_1, e_2}, e_1 :: e_2, \Head e, \Tail e,\Share e, e_1 := e_2$}
    {Preservation has been proven for these cases by \citet{DBLP:books/daglib/0005958}.}


  \case
    {$\userule{E-Enter}$}
    {Evaluation does not alter the expression, therefore this case holds trivially.}

  \case
     {$\userule{E-Update}$}
     {Given that $\Gamma,\Sigma\infers\Update e:\Task \tau$,\refrule{T-Update} gives us that $\Gamma,\Sigma\infers e:\tau$.
     The induction hypothesis gives us that $e\eval v$ also preserves, and thus $\Gamma,\Sigma\infers v:\tau$.
     Therefore $\Gamma,\Sigma\infers\Update v:\Task\tau$.}



  \case
     {$\userule{E-View}$}
     {Given that $\Gamma,\Sigma\infers \View e:\Task \tau$, \refrule{T-View} gives us that $\Gamma,\Sigma\infers e:\tau$.
     The induction hypothesis gives us that $e\eval v$ also preserves, and thus $\Gamma,\Sigma\infers v:\tau$.
     Therefore $\Gamma,\Sigma\infers\View v:\Task\tau$.}

  \case
    {$\userule{E-Pick}$}
    {Given that $\Gamma,\Sigma\infers \Pick\set{\more{L_i \mapsto e_i}} :\Task \tau$, \refrule{T-Pick} gives us that for each $i$ $\Gamma\Sigma\infers e_i:\Task\tau$.
    The induction hypothesis gives us that for each $i$ $e_i\eval t_i$ also preserves, and thus for each $i$ $\Gamma,\Sigma\infers t_i:\Task\tau$.
    Therefore $\Gamma,\Sigma\infers \Pick\set{\more{L_i \mapsto t_i}} :\Task \tau$.}

  \case
    {$\userule{E-Trans}$}
    {}


  \case
     {$\userule{E-Step}$}{}
  %   {Given that $\Gamma,\Sigma\infers e_1\Then e_2:\Task \tau$ and $\Gamma,\Sigma\infers s$, \refrule{T-Then} gives us that $\Gamma,\Sigma\infers e_1:\Task\tau_1$ and $\Gamma,\Sigma\infers e_2:\tau_1 \to \Task \tau$.
  %   By the induction hypothesis, we know that $e_1,s\evaluate t_1,s'$ preserves and thus $\Gamma,\Sigma\infers t_1:\Task\tau_1$ and $\Gamma,\Sigma\infers s'$.
  %   Therefore $\Gamma,\Sigma\infers t_1\Then e_2:\Task\tau$.}

  \case
    {$\userule{E-Forever}$}
    {}

  \case
    {$\userule{E-Done}$}
    {}

  \case
    {$\userule{E-Pair}$}{}
    %   {Given that $\Gamma,\Sigma\infers e_1\And e_2:\Task(\tau_1\times\tau_2)$ and $\Gamma,\Sigma\infers s$, \refrule{T-And} gives us that $\Gamma,\Sigma\infers e_1:\Task\tau_1$ and $\Gamma,\Sigma\infers e_2:\Task\tau_2$.
    %   By the induction hypothesis, we know that both $e_1,s\evaluate t_1,s'$ and $e_2,s'\evaluate t_2,s''$ preserve and thus $\Gamma,\Sigma\infers t_1:\Task\tau_1$, $\Gamma,\Sigma\infers s'$, $\Gamma,\Sigma\infers t_2:\Task\tau_2$ and $\Gamma,\Sigma\infers s''$.
    %   Therefore $\Gamma,\Sigma\infers t_1\And t_2:\Task(\tau_1\times\tau_2)$.}


  \case
     {$\userule{E-Choose}$}
     {Given that $\Gamma,\Sigma\infers e_1\Choose e_2:\Task\tau$, \refrule{T-Choose} gives us that $\Gamma,\Sigma\infers e_1:\Task\tau$ and $\Gamma,\Sigma\infers e_2:\Task\tau$.
     By the induction hypothesis, we have that both $e_1\eval t_1$ and $e_2\eval t_2$ preserve and thus $\Gamma,\Sigma\infers t_1:\Task\tau$ and $\Gamma,\Sigma\infers t_2:\Task\tau$.
     Therefore $\Gamma,\Sigma\infers t_1\Choose t_2:\Task\tau$.}

  \case
    {$\userule{E-Share}$}
    {}

  \case
    {$\userule{E-Assign}$}
    {}

  \case
    {$\userule{E-Change}$}
    {}

  \case
    {$\userule{E-Watch}$}
    {}
\end{proof}
