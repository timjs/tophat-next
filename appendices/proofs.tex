% !TEX root=../main.tex

\section{Proofs}

\subsection{\cref{thm:pres-eval}}

\begin{proof}
  We prove \cref{thm:pres-eval} by induction on $e$:

  \case
    {$e=v,e_1 e_2,\If{e_1}{e_2}{e_3},\tuple{e_1, e_2},\Fst\tuple{e_1, e_2},\Snd\tuple{e_1, e_2}, e_1 :: e_2, \Head e, \Tail e,\Share e, e_1 := e_2$}
    {Preservation has been proven for these cases by \citet{DBLP:books/daglib/0005958}.}


  \case
    {$\userule{E-Enter}$}
    {Evaluation does not alter the expression, therefore this case holds trivially.}

  \case
     {$\userule{E-Update}$}
     {Given that $\Gamma,\Sigma\infers\Update e:\Task \tau$,\refrule{T-Update} gives us that $\Gamma,\Sigma\infers e:\tau$.
     The induction hypothesis gives us that $e\eval v$ also preserves, and thus $\Gamma,\Sigma\infers v:\tau$.
     Therefore $\Gamma,\Sigma\infers\Update v:\Task\tau$.}



  \case
     {$\userule{E-View}$}
     {Given that $\Gamma,\Sigma\infers \View e:\Task \tau$, \refrule{T-View} gives us that $\Gamma,\Sigma\infers e:\tau$.
     The induction hypothesis gives us that $e\eval v$ also preserves, and thus $\Gamma,\Sigma\infers v:\tau$.
     Therefore $\Gamma,\Sigma\infers\View v:\Task\tau$.}

  \case
    {$\userule{E-Pick}$}
    {Given that $\Gamma,\Sigma\infers \Pick\set{\more{L_i \mapsto e_i}} :\Task \tau$, \refrule{T-Pick} gives us that for each $i$ $\Gamma,\Sigma\infers e_i:\Task\tau$.
    The induction hypothesis gives us that for each $i$ $e_i\eval t_i$ also preserves, and thus for each $i$ $\Gamma,\Sigma\infers t_i:\Task\tau$.
    Therefore $\Gamma,\Sigma\infers \Pick\set{\more{L_i \mapsto t_i}} :\Task \tau$.}

  \case
    {$\userule{E-Trans}$}
    {Given that $\Gamma,\Sigma\infers e_1\Trans e_2:Task\tau_2$, \refrule{T-Trans} gives us that $\Gamma,\Sigma\infers e_1:\tau_1\to\tau_2$ and $\Gamma,\Sigma\infers e_2:\Task \tau_1$.
    The induction hypothesis gives us that $e_2\eval t_2$ also preserves, and thus $\Gamma,\Sigma\infers t_2:\Task\tau_1$.
    Therefore $\Gamma,\Sigma\infers e_1\Trans t_2:\Task\tau_2$.}


  \case
     {$\userule{E-Step}$}{}
     {Given that $\Gamma,\Sigma\infers e_1\Step e_2:Task\tau_2$, \refrule{T-Step} gives us that $\Gamma,\Sigma\infers e_1:\Task\tau_1$ and $\Gamma,\Sigma\infers e_2:\tau_1\to\Task \tau_2$.
     The induction hypothesis gives us that $e_1\eval t_1$ also preserves, and thus $\Gamma,\Sigma\infers t_1:\Task\tau_1$.
     Therefore $\Gamma,\Sigma\infers t_1\Step e_2:\Task\tau_2$.}

  \case
    {$\userule{E-Forever}$}
    {Given that $\Gamma,\Sigma\infers \Forever e : \Task \Void$, \refrule{T-Forever} gives us that $\Gamma,\Sigma\infers e:\Task \tau$.
    The induction hypothesis gives us that $e\eval t$ also preserves, and thus $\Gamma,\Sigma\infers t:\Task \tau$.
    Therefore $\Gamma,\Sigma\infers \Forever t : \Task \Void$\todo{something strange is going on here. Type information is deleted, and therefore, forever always preserves (since result type is void)}}

  \case
    {$\userule{E-Done}$}
    {Given that $\Gamma,\Sigma\infers\Done e :\Task\tau$, \refrule{T-Done} gives us that $\Gamma,\Sigma \infers e:\tau$.
    The induction hypothesis gives us that $e\eval v$ also preverves, and thus $\Gamma,\Sigma\infers v:\tau$.
    Therefore $\Gamma,\Sigma\infers \Done v : \Task \tau$.}

  \case
    {$\userule{E-Pair}$}
    {Give that $\Gamma,\Sigma\infers e_1\Pair e_2:\Task(\tau_1\times\tau_2)$, \refrule{T-Pair} gives us that $\Gamma,\Sigma\infers e_1:\Task\tau_1$ and $\Gamma,\Sigma\infers e_2:\Task\tau_2$.
    By the induction hypothesis, we know that both $e_1\eval t_1$ and $e_2\eval t_2$ preserve and thus $\Gamma,\Sigma\infers t_1:\Task\tau_1$ and $\Gamma,\Sigma\infers t_2:\Task\tau_2$.
    Therefore $\Gamma,\Sigma\infers t_1\Pair t_2:\Task(\tau_1\times\tau_2)$.}


  \case
     {$\userule{E-Choose}$}
     {Given that $\Gamma,\Sigma\infers e_1\Choose e_2:\Task\tau$, \refrule{T-Choose} gives us that $\Gamma,\Sigma\infers e_1:\Task\tau$ and $\Gamma,\Sigma\infers e_2:\Task\tau$.
     By the induction hypothesis, we have that both $e_1\eval t_1$ and $e_2\eval t_2$ preserve and thus $\Gamma,\Sigma\infers t_1:\Task\tau$ and $\Gamma,\Sigma\infers t_2:\Task\tau$.
     Therefore $\Gamma,\Sigma\infers t_1\Choose t_2:\Task\tau$.}

  \case
    {$\userule{E-Share}$}
    {Given that $\Gamma,\Sigma\infers \Share e : \Task\ (\Reference \beta)$, \refrule{T_Share} gives us that $\Gamma,\Sigma\infers e:\beta$.
    By the induction hypothesis, we have that $e\eval v$ also preserves and thus $\Gamma,\Sigma\infers v:\beta$.
    Therefore $\Gamma,\Sigma\infers \Share v:\Task \beta$.}

  \case
    {$\userule{E-Assign}$}
    {Given that $\Gamma,\Sigma\infers e_1 := e_2 :\Task \Unit$, \refrule{T-Share} gives us that $\Gamma,\Sigma\infers e_1:\Reference \beta$ and $\Gamma,\Sigma\infers e_2:\beta$.
    By the induction hypothesis, we have that $e_1\eval l$ and $e_2\eval v$ also preserve and thus $\Gamma,\Sigma\infers l:\Reference \beta$ and $\Gamma,\Sigma\infers v:\beta$.
    Therefore $\Gamma,\Sigma\infers l:=v:\Task \Unit$.\todo{again, type is being erased} }

  \case
    {$\userule{E-Change}$}
    {Given that $\Gamma,\Sigma\infers \Change e :\Task \beta$, \refrule{T-Change} gives us that $\Gamma,\Sigma\infers e:\Reference \beta$.
    By the induction hypothesis, we have that $e\eval l$ also preserves and thus $\Gamma,\Sigma\infers l:\Reference \beta$.
    Therefore $\Gamma,\Sigma\infers \Change l:\Task\beta$.}

  \case
    {$\userule{E-Watch}$}
    {Given that $\Gamma,\Sigma\infers \Watch e :\Task \beta$, \refrule{T-Watch} gives us that $\Gamma,\Sigma\infers e:\Reference \beta$.
    By the induction hypothesis, we have that $e\eval l$ also preserves and thus $\Gamma,\Sigma\infers l:\Reference \beta$.
    Therefore $\Gamma,\Sigma\infers \Watch l:\Task\beta$.}
\end{proof}


\subsection{\cref{thm:pres-norm}}

We prove \cref{thm:pres-norm} by induction on $e$:

\case
  {$\userule{N-Done}$,$\userule{N-Enter}$,$\userule{N-Update}$,$\userule{N-View}$,\\$\userule{N-Pick}$,$\userule{N-Change}$,$\userule{N-Watch}$,$\userule{N-Fail}$}
  {In these cases, the expression and state remain unaltered, and therefore the lemma holds trivially.}

\case
  {$\userule{N-StepNone}$,$\userule{N-StepFail}$,\\$\userule{N-StepWait}$}
  {The proof of these three cases is identical, namely as follows.\\
  Given that $\Gamma,\Sigma\infers t_1\Step e_2:\Task\tau_2$ and $\Gamma,\Sigma\infers \sigma$, \refrule{T-Step} gives us that $\Gamma,\Sigma\infers t_1:\Task\tau_1$ and $\Gamma,\Sigma\infers e_2:\tau_1\to\Task \tau_2$.
  The induction hypothesis gives us that $t_1,\sigma\normalise t_1',\sigma',\delta'$ also preserves, and thus $\Gamma,\Sigma\infers t_1':\Task\tau_1$ and $\Gamma,\Sigma\infers\sigma'$.
  Therefore $\Gamma,\Sigma\infers t_1'\Step e_2:\Task\tau_2$.}

\case
  {$\userule{N-StepCont}$}
  {Given that $\Gamma,\Sigma\infers t_1\Step e_2:\Task\tau_2$ and $\Gamma,\Sigma\infers \sigma$, \refrule{T-Step} gives us that $\Gamma,\Sigma\infers t_1:\Task\tau_1$ and $\Gamma,\Sigma\infers e_2:\tau_1\to\Task \tau_2$.
  The induction hypothesis gives us that $t_1,\sigma\normalise t_1',\sigma',\delta'$ also preserves, and thus $\Gamma,\Sigma\infers t_1':\Task\tau_1$ and $\Gamma,\Sigma\infers\sigma'$.
  \cref{lem:pres-value} gives us that $\Value{t_1',\sigma'}=v_1$ preserves, and therefore $\Gamma,\Sigma\infers v_1:\tau_1$.
  \refrule{T-App} gives us that $\Gamma,\Sigma\infers e_2 v_1 : \Task\tau_2$.
  From \cref{thm:pres-eval} we get that $\Gamma,\Sigma\infers t_2:\Task\tau_1$.
  Then by applying the induction hypothesis a second time, we get that $t_1\sigma'\normalise t_2',\sigma'',\delta''$ also preserves and thus $\Gamma,\Sigma\infers t_2':\Task\tau_1$ and $\Gamma,\Sigma\infers \sigma''$, which is exactly what we needed to show.}

\case
  {$\userule{N-ChooseLeft}$}
  {Given that $\Gamma,\Sigma\infers t_1 \Choose t_2 : \Task \tau$ and $\Gamma,\Sigma\infers\sigma$, \refrule{T-Choose} gives us that $\Gamma,\Sigma\infers t_1:\Task\tau$.
  The induction hypothesis gives us that $t_1,\sigma\normalise t_1',\sigma',\delta'$ also preserves,
  and thus $\Gamma,\Sigma\infers t_1':\Task\tau$ and $\Gamma,\Sigma\infers \sigma'$, which is exactly what we needed to show.}

\case
  {$\userule{N-ChooseRight}$}
  {Given that $\Gamma,\Sigma\infers t_1 \Choose t_2 : \Task \tau$ and $\Gamma,\Sigma\infers\sigma$, \refrule{T-Choose} gives us that $\Gamma,\Sigma\infers t_1:\Task\tau$ and $\Gamma,\Sigma\infers t_2:\Task \tau$.
  The induction hypothesis gives us that $t_1,\sigma\normalise t_1',\sigma',\delta'$ also preserves,
  and thus $\Gamma,\Sigma\infers t_1':\Task\tau$ and $\Gamma,\Sigma\infers \sigma'$.
  A second application of the induction hypothesis gives us that $t_2,\sigma'\normalise t_2',\sigma'',\delta''$ also preserves, and thus $\Gamma,\Sigma\infers t_2':\Task\tau$ and $\Gamma,\Sigma\infers\sigma''$, which is exactly what we needed to show.}

\case
  {$\userule{N-ChooseNone}$}
  {Given that $\Gamma,\Sigma\infers t_1 \Choose t_2 : \Task \tau$ and $\Gamma,\Sigma\infers\sigma$, \refrule{T-Choose} gives us that $\Gamma,\Sigma\infers t_1:\Task\tau$ and $\Gamma,\Sigma\infers t_2:\Task \tau$.
  The induction hypothesis gives us that $t_1,\sigma\normalise t_1',\sigma',\delta'$ also preserves,
  and thus $\Gamma,\Sigma\infers t_1':\Task\tau$ and $\Gamma,\Sigma\infers \sigma'$.
  A second application of the induction hypothesis gives us that $t_2,\sigma'\normalise t_2',\sigma'',\delta''$ also preserves, and thus $\Gamma,\Sigma\infers t_2':\Task\tau$ and $\Gamma,\Sigma\infers\sigma''$.
  Therefore $\Gamma,\Sigma\infers t_1'\Choose t_2':\Task\tau$.}

\case
  {$\userule{N-Trans}$}
  {Given that $\Gamma,\Sigma\infers e_1\Trans t_2:\Task\tau_2$ and $\Gamma,\Sigma\infers\sigma$, \refrule{T-Trans} gives us that $\Gamma,\Sigma\infers e_1:\tau_1\to \tau_2$ and $\Gamma,\Sigma\infers t_2:\Task\tau_1$.
  The induction hypothesis gives us that $t_2,\sigma\normalise t_2',\sigma',\delta'$ also preserves, and thus $\Gamma,\Sigma\infers t_2':\Task\tau_1$ and $\Gamma,\Sigma\infers\sigma'$.
  Therefore $\Gamma,\Sigma\infers e_1\Trans t_2':\Task\tau_2$.}

\case
  {$\userule{N-Forever}$}
  {\todo{Not sure what to do here, since $t'$ is NOT smaller, so can we apply induction?}}

\case
  {$\userule{N-Pair}$}
  {Given that $\Gamma,\Sigma\infers t_1\Pair t_2:\Task (\tau_1\times\tau_2)$ and $\Gamma,\Sigma\infers\sigma$, \refrule{T-Pair} gives us that $\Gamma,\Sigma\infers t_1:\Task \tau_1$ and $\Gamma,\Sigma\infers t_2:\Task \tau_2$.
  The induction hypothesis gives us that $t_1,\sigma\normalise t_1',\sigma',\delta'$ also preserves,
  and thus $\Gamma,\Sigma\infers t_1'$ and $\Gamma,\Sigma\infers \sigma'$.
  A second application of the induction hypothesis gives us that $t_2,\sigma'\normalise t_2',\sigma'',\delta''$ also preserves, and thus $\Gamma,\Sigma\infers t_2':\Task\tau_2$ and $\Gamma,\Sigma\infers \sigma''$.
  Therefore $\Gamma,\Sigma\infers t_1'\Pair t_2':\Task (\tau_1\times\tau_2)$.}

\case
  {$\userule{N-Share}$}
  {Given that $\Gamma,\Sigma\infers \Share b:\Task\ (\Reference \beta)$ and $\Gamma,\Sigma\infers\sigma$.
  ?????? \todo{fix}
  }

\case
  {$\userule{N-Assign}$}
  {}


\subsection{\cref{thm:pres-fix}}

\case
  {$\userule{F-Done}$}
  {Given that $\Gamma,\Sigma\infers t :\Task\tau$ and $\Gamma,\Sigma\infers \sigma$, we obtain by \cref{thm:pres-norm} that $t,\sigma\normalise t',\sigma',\delta'$ preserves. Therefore $\Gamma,\Sigma\infers t':\Task\tau$ and $\Gamma,\Sigma\infers\sigma'$, which is exactly what we need to show in this case.}

\case
  {$\userule{F-Loop}$}
  {Given that $\Gamma,\Sigma\infers t :\Task\tau$ and $\Gamma,\Sigma\infers \sigma$, we obtain by \cref{thm:pres-norm} that $t,\sigma\normalise t',\sigma',\delta'$ preserves. Therefore $\Gamma,\Sigma\infers t':\Task\tau$ and $\Gamma,\Sigma\infers\sigma'$.
  By application of the induction hypothesis, we obtain that $t',\sigma',\delta'\fix t'',\sigma''$ also preserves, and thus $\Gamma,\Sigma\infers t'':\Task\tau$ and $\Gamma,\Sigma\infers\sigma''$, which is exactly what we need to show in this case.}
